%% -!TEX root = AUTthesis.tex
% در این فایل، عنوان پایان‌نامه، مشخصات خود، متن تقدیمی‌، ستایش، سپاس‌گزاری و چکیده پایان‌نامه را به فارسی، وارد کنید.
% توجه داشته باشید که جدول حاوی مشخصات پروژه/پایان‌نامه/رساله و همچنین، مشخصات داخل آن، به طور خودکار، درج می‌شود.
%%%%%%%%%%%%%%%%%%%%%%%%%%%%%%%%%%%%
% دانشکده، آموزشکده و یا پژوهشکده  خود را وارد کنید
\faculty{دانشکده مهندسی کامپیوتر}
% گرایش و گروه آموزشی خود را وارد کنید
 \department{}
% عنوان پایان‌نامه را وارد کنید
% \fatitle{
% 	درس داده‌کاوی
% }
% نام استاد(ان) راهنما را وارد کنید
% \firstsupervisor{دکتر مریم امیر مزلقانی}
%\secondsupervisor{استاد راهنمای دوم}
% نام استاد(دان) مشاور را وارد کنید. چنانچه استاد مشاور ندارید، دستور پایین را غیرفعال کنید.
% \firstadvisor{دکتر حامد فربه}
%\secondadvisor{استاد مشاور دوم}
% نام نویسنده را وارد کنید
% \name{امیرحسین پاشایی‌هیر، حیدر فهمی، مهدی قیاسی }
% نام خانوادگی نویسنده را وارد کنید
\surname{}
%%%%%%%%%%%%%%%%%%%%%%%%%%%%%%%%%%
% \thesisdate{زمستان ۱۴۰۰}

% چکیده پایان‌نامه را وارد کنید
\fa-abstract{
در این گزارش به بررسی روشی به نام تی اس ان ای می‌پردازیم که هدف آن مصور سازی داده‌هایی با ابعاد بالا می‌باشد که این هدف را با نگاشت کردن هر داده به یک نقطه در فضای دو یا سه بعدی انجام می‌دهد. این روش یک نسخه از روش تعبیه همسایه تصادفی (یا اس ان ای) می‌باشد که بهینه کردن آن بسیار راحت‌تر بوده و با کم کردن گرایش نقاط متراکم به مرکز، خروجی به مراتب بهتری تحویل می‌دهد. تی اس ان ای از روش‌های موجود، عملکرد بهتری در ارائه انواع ساختارهای داده کلی در قالب یک نگاشت، ارائه می‌دهد. این عملکرد بخصوص برای داده‌هایی که ابعاد بالایی دارند اما در اصل به روی چند منحی با ابعادی پایین قرار داده شده‌اند بسیار مهم می‌شود. برای مصور سازی ساختار داده‌های با ابعاد بالا، بررسی می‌کنیم که تی اس ان ای چگونه با استفاده از قدم زدن اتفاقی به روی گراف همسایگی می‌تواند ساختار داده‌ها را خلاصه کند و آن‌ها را مصور سازی کند، ما همچنین به مقایسه‌ی عملکرد روش تی اس ان ای نسبت به تعداد زیادی از روش‌ها مانند نگاشت سامون، ایزومپ و تعبیه خطی محلی می‌پردازیم و در نتیجه می‌بینم که تی اس ان ای، مصور سازی به مراتب بهتری نسبت به بقیه روش‌ها بر روی تمام مجموعه داده‌ها ارائه داده است.
}


% کلمات کلیدی پایان‌نامه را وارد کنید
\keywords{مصورسازی، کاهش بعد، الگوریتم‌های تعبیه سازی، مقیاس چندبعدی}



\AUTtitle
%%%%%%%%%%%%%%%%%%%%%%%%%%%%%%%%%%
% \vspace*{7cm}
% \thispagestyle{empty}
% \begin{center}
% \includegraphics[height=5cm,width=12cm]{besm}
% \end{center}